% Manuelles Kerning für bestimmte Kombinationen
\XeTeXinterchartokenstate = 1
\newXeTeXintercharclass \charHalbgeviert
\newXeTeXintercharclass \charNummerEins
\newXeTeXintercharclass \charNummerZwei
\newXeTeXintercharclass \charNummerDrei
\newXeTeXintercharclass \charNummerVier
\newXeTeXintercharclass \charNummerFünf
\newXeTeXintercharclass \charNummerSechs
\newXeTeXintercharclass \charNummerSieben
\newXeTeXintercharclass \charNummerAcht
\newXeTeXintercharclass \charNummerNeun
\newXeTeXintercharclass \charNummerNull

\newXeTeXintercharclass \charKomma
\newXeTeXintercharclass \charPunkt
\newXeTeXintercharclass \charSlash
\newXeTeXintercharclass \charParens
\newXeTeXintercharclass \charDivis
\newXeTeXintercharclass \charKzg
\newXeTeXintercharclass \charColon

\newXeTeXintercharclass \chara
\newXeTeXintercharclass \charb
\newXeTeXintercharclass \charc
\newXeTeXintercharclass \chard
\newXeTeXintercharclass \chare
\newXeTeXintercharclass \charf
\newXeTeXintercharclass \charg
\newXeTeXintercharclass \charh
\newXeTeXintercharclass \chari
\newXeTeXintercharclass \chark
\newXeTeXintercharclass \charl
\newXeTeXintercharclass \charm
\newXeTeXintercharclass \charn
\newXeTeXintercharclass \charo
\newXeTeXintercharclass \charp
\newXeTeXintercharclass \charr
\newXeTeXintercharclass \chars
\newXeTeXintercharclass \chart
\newXeTeXintercharclass \charu
\newXeTeXintercharclass \charv
\newXeTeXintercharclass \charx
\newXeTeXintercharclass \charz

\newXeTeXintercharclass \charA
\newXeTeXintercharclass \charD
\newXeTeXintercharclass \charE
\newXeTeXintercharclass \charG
\newXeTeXintercharclass \charH
\newXeTeXintercharclass \charI
\newXeTeXintercharclass \charJ
\newXeTeXintercharclass \charK
\newXeTeXintercharclass \charM
\newXeTeXintercharclass \charP
\newXeTeXintercharclass \charS
\newXeTeXintercharclass \charT
\newXeTeXintercharclass \charV
\newXeTeXintercharclass \charW
\newXeTeXintercharclass \charX
\newXeTeXintercharclass \charZ

\XeTeXcharclass `– \charHalbgeviert
\XeTeXcharclass `1 \charNummerEins
\XeTeXcharclass `2 \charNummerZwei
\XeTeXcharclass `3 \charNummerDrei
\XeTeXcharclass `4 \charNummerVier
\XeTeXcharclass `5 \charNummerFünf
\XeTeXcharclass `6 \charNummerSechs
\XeTeXcharclass `7 \charNummerSieben
\XeTeXcharclass `8 \charNummerAcht
\XeTeXcharclass `9 \charNummerNeun
\XeTeXcharclass `0 \charNummerNull

\XeTeXcharclass `, \charKomma
\XeTeXcharclass `. \charPunkt
\XeTeXcharclass `/ \charSlash
\XeTeXcharclass `( \charParens	
\XeTeXcharclass `) \charParens
\XeTeXcharclass `: \charColon
\XeTeXcharclass `[ \charParens
\XeTeXcharclass `- \charDivis
\XeTeXcharclass `ˈ \charKzg

\XeTeXcharclass `a \chara
\XeTeXcharclass `b \charb
\XeTeXcharclass `c \charc
\XeTeXcharclass `d \chard
\XeTeXcharclass `e \chare
\XeTeXcharclass `f \charf
\XeTeXcharclass `g \charg
\XeTeXcharclass `h \charh
\XeTeXcharclass `i \chari
\XeTeXcharclass `k \chark
\XeTeXcharclass `l \charl
\XeTeXcharclass `m \charm
\XeTeXcharclass `n \charn
\XeTeXcharclass `o \charo
\XeTeXcharclass `p \charp
\XeTeXcharclass `r \charr
\XeTeXcharclass `s \chars
\XeTeXcharclass `t \chart
\XeTeXcharclass `u \charu
\XeTeXcharclass `v \charv
\XeTeXcharclass `x \charx
\XeTeXcharclass `z \charz

\XeTeXcharclass `A \charA
\XeTeXcharclass `D \charD
\XeTeXcharclass `E \charE
\XeTeXcharclass `G \charG
\XeTeXcharclass `H \charH
\XeTeXcharclass `I \charI
\XeTeXcharclass `J \charJ
\XeTeXcharclass `K \charK
\XeTeXcharclass `M \charM
\XeTeXcharclass `P \charP
\XeTeXcharclass `S \charS
\XeTeXcharclass `T \charT
\XeTeXcharclass `V \charV
\XeTeXcharclass `W \charW
\XeTeXcharclass `X \charX
\XeTeXcharclass `Z \charZ

% Ausgleich um Halbgeviert herum
\XeTeXinterchartoks \charNummerEins \charHalbgeviert {\kern 0.3ex}% 2
\XeTeXinterchartoks \charHalbgeviert \charNummerEins {\kern 0.25ex}
\XeTeXinterchartoks \charNummerZwei \charHalbgeviert {\kern 0.2ex}% 2
\XeTeXinterchartoks \charHalbgeviert \charNummerZwei {\kern 0.25ex}
\XeTeXinterchartoks \charNummerDrei \charHalbgeviert {\kern 0.2ex}% 3
\XeTeXinterchartoks \charHalbgeviert \charNummerDrei {\kern 0.25ex}
\XeTeXinterchartoks \charNummerVier \charHalbgeviert {\kern 0.25ex}% 4
\XeTeXinterchartoks \charHalbgeviert \charNummerVier {\kern 0.2ex}%{\kern 0.425ex}
\XeTeXinterchartoks \charNummerFünf \charHalbgeviert {\kern 0.2ex}% 5
\XeTeXinterchartoks \charHalbgeviert \charNummerFünf {\kern 0.25ex}
\XeTeXinterchartoks \charNummerSechs \charHalbgeviert {\kern 0.21ex}% 6
\XeTeXinterchartoks \charHalbgeviert \charNummerSieben {\kern 0.3ex}
\XeTeXinterchartoks \charNummerSieben \charHalbgeviert {\kern 0.3ex}% 6
\XeTeXinterchartoks \charHalbgeviert \charNummerSechs {\kern 0.35ex}
\XeTeXinterchartoks \charNummerAcht \charHalbgeviert {\kern 0.25ex}% 8
\XeTeXinterchartoks \charHalbgeviert \charNummerAcht {\kern 0.25ex}
\XeTeXinterchartoks \charNummerNeun \charHalbgeviert {\kern 0.2ex}% 9
\XeTeXinterchartoks \charHalbgeviert \charNummerNeun {\kern 0.2ex}
\XeTeXinterchartoks \charNummerNull \charHalbgeviert {\kern 0.3ex}% 0
\XeTeXinterchartoks \charHalbgeviert \charNummerNull {\kern 0.35ex}

\XeTeXinterchartoks \charHalbgeviert \charE {\kern 0.25ex}

% Ausgleich um / herum
\XeTeXinterchartoks \charNummerEins \charSlash {\kern 0.3ex}% 2
\XeTeXinterchartoks \charSlash \charNummerEins {\kern 0.1ex}
\XeTeXinterchartoks \charNummerZwei \charSlash {\kern 0.2ex}% 2
\XeTeXinterchartoks \charSlash \charNummerZwei {\kern 0.1ex}
\XeTeXinterchartoks \charNummerDrei \charSlash {\kern 0.2ex}% 3
\XeTeXinterchartoks \charSlash \charNummerDrei {\kern 0.15ex}
\XeTeXinterchartoks \charNummerVier \charSlash {\kern 0.3ex}% 4
%\XeTeXinterchartoks \charSlash \charNummerVier {\kern 0.05ex}%{\kern 0.425ex}
\XeTeXinterchartoks \charNummerFünf \charSlash {\kern 0.25ex}% 5
\XeTeXinterchartoks \charSlash \charNummerFünf {\kern 0.1ex}
\XeTeXinterchartoks \charNummerSechs \charSlash {\kern 0.3ex}% 6
\XeTeXinterchartoks \charSlash \charNummerSechs {\kern 0.15ex}
\XeTeXinterchartoks \charNummerSieben \charSlash {\kern 0.1ex}% 6
\XeTeXinterchartoks \charSlash \charNummerSieben {\kern 0.3ex}
\XeTeXinterchartoks \charNummerAcht \charSlash {\kern 0.2ex}% 8
\XeTeXinterchartoks \charSlash \charNummerAcht {\kern 0.25ex}
\XeTeXinterchartoks \charNummerNeun \charSlash {\kern 0.2ex}% 9
%\XeTeXinterchartoks \charSlash \charNummerNeun {\kern 0.05ex}
\XeTeXinterchartoks \charNummerNull \charSlash {\kern 0.3ex}% 0
%\XeTeXinterchartoks \charSlash \charNummerNull {\kern 0.1ex}

\XeTeXinterchartoks \charSlash \charSlash {\kern -0.3ex}% 0

\XeTeXinterchartoks \chara \charSlash {\kern 0.3ex}% 0
\XeTeXinterchartoks \charb \charSlash {\kern 0.25ex}% 0
\XeTeXinterchartoks \chard \charSlash {\kern 0.25ex}% 0
\XeTeXinterchartoks \chare \charSlash {\kern 0.25ex}% 0
\XeTeXinterchartoks \charf \charSlash {\kern 0.2ex}% 0
\XeTeXinterchartoks \charg \charSlash {\kern 0.25ex}% 0
\XeTeXinterchartoks \charh \charSlash {\kern 0.25ex}%
\XeTeXinterchartoks \chari \charSlash {\kern 0.2ex}%
\XeTeXinterchartoks \chark \charSlash {\kern 0.3ex}% 0
\XeTeXinterchartoks \charl \charSlash {\kern 0.2ex}% 0
\XeTeXinterchartoks \charm \charSlash {\kern 0.25ex}% 0
\XeTeXinterchartoks \charn \charSlash {\kern 0.25ex}% 0
\XeTeXinterchartoks \charo \charSlash {\kern 0.25ex}% 0
\XeTeXinterchartoks \charp \charSlash {\kern 0.25ex}% 0
\XeTeXinterchartoks \charr \charSlash {\kern 0.15ex}% 0
\XeTeXinterchartoks \chars \charSlash {\kern 0.25ex}% 0
\XeTeXinterchartoks \chart \charSlash {\kern 0.325ex}% 0
\XeTeXinterchartoks \charu \charSlash {\kern 0.25ex}% 0
\XeTeXinterchartoks \charx \charSlash {\kern 0.2ex}% 0
\XeTeXinterchartoks \charz \charSlash {\kern 0.25ex}% 0

\XeTeXinterchartoks \charSlash \charb {\kern 0.3ex}
\XeTeXinterchartoks \charSlash \charf {\kern 0.175ex}
\XeTeXinterchartoks \charSlash \chari {\kern 0.15ex}

\XeTeXinterchartoks \charI \charSlash {\kern 0.2ex}%

\XeTeXinterchartoks \charSlash \charD {\kern 0.15ex}
\XeTeXinterchartoks \charSlash \charE {\kern 0.25ex}
\XeTeXinterchartoks \charSlash \charG {\kern 0.25ex}
\XeTeXinterchartoks \charSlash \charH {\kern 0.25ex}
\XeTeXinterchartoks \charSlash \charJ {\kern 0.25ex}
\XeTeXinterchartoks \charSlash \charK {\kern 0.3ex}
\XeTeXinterchartoks \charSlash \charM {\kern 0.25ex}
\XeTeXinterchartoks \charSlash \charS {\kern 0.25ex}
\XeTeXinterchartoks \charSlash \charT {\kern 0.25ex}
\XeTeXinterchartoks \charSlash \charV {\kern 0.3ex}
\XeTeXinterchartoks \charSlash \charW {\kern 0.25ex}
\XeTeXinterchartoks \charSlash \charZ {\kern 0.2ex}

% Doppelpunkt
\XeTeXinterchartoks \charColon \charSlash {\kern 0.1ex}% 0
\XeTeXinterchartoks \charColon \chard {\kern 0.15ex}% 0

\XeTeXinterchartoks \charz \charColon {\kern 0.3ex}% 0
\XeTeXinterchartoks \charColon \charNummerNull {\kern 0.075ex}% 0

% Halbgeviert + klein
\XeTeXinterchartoks \charHalbgeviert \charb {\kern 0.25ex}
\XeTeXinterchartoks \charHalbgeviert \charc {\kern 0.25ex}
\XeTeXinterchartoks \charHalbgeviert \chare {\kern 0.2ex}
\XeTeXinterchartoks \charHalbgeviert \charl {\kern 0.2ex}
\XeTeXinterchartoks \charHalbgeviert \charv {\kern 0.25ex}

% klein + Halbgeviert
\XeTeXinterchartoks \chara \charHalbgeviert {\kern 0.25ex}
\XeTeXinterchartoks \charc \charHalbgeviert {\kern 0.25ex}
\XeTeXinterchartoks \charr \charHalbgeviert {\kern 0.3ex}
\XeTeXinterchartoks \charv \charHalbgeviert {\kern 0.3ex}
\XeTeXinterchartoks \charx \charHalbgeviert {\kern 0.2ex}

% Halbgeviert + Groß
\XeTeXinterchartoks \charHalbgeviert \charA {\kern 0.35ex}
\XeTeXinterchartoks \charHalbgeviert \charD {\kern 0.35ex}
\XeTeXinterchartoks \charHalbgeviert \charJ {\kern 0.25ex}
\XeTeXinterchartoks \charHalbgeviert \charP {\kern 0.25ex}
\XeTeXinterchartoks \charHalbgeviert \charS {\kern 0.3ex}

% Groß + Halbgeviert

% Spationieren vor f. in Seitenangaben
\XeTeXinterchartoks \charNummerNull \charf {\kern 0.225ex}
\XeTeXinterchartoks \charNummerEins \charf {\kern 0.2ex}
\XeTeXinterchartoks \charNummerZwei \charf {\kern 0.225ex}
\XeTeXinterchartoks \charNummerDrei \charf {\kern 0.25ex}
\XeTeXinterchartoks \charNummerVier \charf {\kern 0.25ex}
\XeTeXinterchartoks \charNummerFünf \charf {\kern 0.25ex}
\XeTeXinterchartoks \charNummerSechs \charf {\kern 0.25ex}
\XeTeXinterchartoks \charNummerSieben \charf {\kern 0.25ex}
\XeTeXinterchartoks \charNummerAcht \charf {\kern 0.25ex}
\XeTeXinterchartoks \charNummerNeun \charf {\kern 0.25ex}

% vor v in verso
\XeTeXinterchartoks \charNummerEins \charv {\kern 0.1ex}

% Ausgleich mit Klammern
\XeTeXinterchartoks \charParens \charJ {\kern 1pt}

% % Ausgleich vor Komma
% \XeTeXinterchartoks \charNummerEins \charKomma {\kern -0.2ex}% 1
% \XeTeXinterchartoks \charNummerZwei \charKomma {\kern -0.1ex}% 2
% \XeTeXinterchartoks \charNummerDrei \charKomma {\kern -0.2ex}% 3
\XeTeXinterchartoks \charNummerVier \charKomma {\kern 0.1ex}% 4
% \XeTeXinterchartoks \charNummerFünf \charKomma {\kern -0.2ex}% 5
% \XeTeXinterchartoks \charNummerSechs \charKomma {\kern -0.1ex}% 6
% \XeTeXinterchartoks \charNummerSieben \charKomma {\kern -0.2ex}% 7
% \XeTeXinterchartoks \charNummerAcht \charKomma {\kern -0.1ex}% 8
% \XeTeXinterchartoks \charNummerNeun \charKomma {\kern -0.2ex}% 9
% \XeTeXinterchartoks \charNummerNull \charKomma {\kern -0.1ex}% 0
% 
% %Ausgleich nach Komma
\XeTeXinterchartoks \charKomma \charNummerEins {\kern 0.2ex}%1
\XeTeXinterchartoks \charKomma \charNummerZwei {\kern 0.2ex}
\XeTeXinterchartoks \charKomma \charNummerDrei {\kern 0.15ex}
\XeTeXinterchartoks \charKomma \charNummerVier {\kern 0.2ex}
\XeTeXinterchartoks \charKomma \charNummerFünf {\kern 0.0ex}%5
\XeTeXinterchartoks \charKomma \charNummerSechs {\kern 0.0ex}
\XeTeXinterchartoks \charKomma \charNummerSieben {\kern 0.0ex}
\XeTeXinterchartoks \charKomma \charNummerAcht {\kern 0.1ex}
\XeTeXinterchartoks \charKomma \charNummerNeun {\kern 0.0ex}
\XeTeXinterchartoks \charKomma \charNummerNull {\kern 0.0ex}%0

\XeTeXinterchartoks \charKomma \chara {\kern 0.1ex}%0
\XeTeXinterchartoks \charKomma \charb {\kern 0.2ex}%0

\XeTeXinterchartoks \charKomma \charA {\kern 0.3ex}%0
\XeTeXinterchartoks \charKomma \charI {\kern 0.4ex}%0
\XeTeXinterchartoks \charKomma \charV {\kern 0.3ex}%0
\XeTeXinterchartoks \charKomma \charX {\kern 0.4ex}%0
% 
% % Ausgleich vor Punkt
\XeTeXinterchartoks \charNummerEins \charPunkt {\kern -0.1ex}% 1
\XeTeXinterchartoks \charNummerZwei \charPunkt {\kern -0.05ex}% 2
\XeTeXinterchartoks \charNummerDrei \charPunkt {\kern -0.1ex}% 3
\XeTeXinterchartoks \charNummerVier \charPunkt {\kern 0.0ex}% 4
\XeTeXinterchartoks \charNummerFünf \charPunkt {\kern -0.2ex}% 5
\XeTeXinterchartoks \charNummerSechs \charPunkt {\kern -0.1ex}% 6
\XeTeXinterchartoks \charNummerSieben \charPunkt {\kern -0.2ex}% 7
\XeTeXinterchartoks \charNummerAcht \charPunkt {\kern -0.05ex}% 8
\XeTeXinterchartoks \charNummerNeun \charPunkt {\kern -0.1ex}% 9
\XeTeXinterchartoks \charNummerNull \charPunkt {\kern -0.1ex}% 0
% 
% %Ausgleich nach Punkt
\XeTeXinterchartoks \charPunkt \charNummerEins {\kern 0.2ex}%1
\XeTeXinterchartoks \charPunkt \charNummerZwei {\kern 0.15ex}
\XeTeXinterchartoks \charPunkt \charNummerDrei {\kern 0.2ex}
\XeTeXinterchartoks \charPunkt \charNummerVier {\kern 0.25ex}
\XeTeXinterchartoks \charPunkt \charNummerFünf {\kern 0.0ex}%5
\XeTeXinterchartoks \charPunkt \charNummerSechs {\kern 0.2ex}
\XeTeXinterchartoks \charPunkt \charNummerSieben {\kern 0.2ex}
\XeTeXinterchartoks \charPunkt \charNummerAcht {\kern 0.2ex}
\XeTeXinterchartoks \charPunkt \charNummerNeun {\kern 0.2ex}
\XeTeXinterchartoks \charPunkt \charNummerNull {\kern 0.2ex}%0
\XeTeXinterchartoks \charPunkt \charHalbgeviert {\kern 0.25ex}%0

% % Ausgleich nach Divis
% \XeTeXinterchartoks \charDivis \charNummerDrei {\kern 0.1ex}%0
% \XeTeXinterchartoks \charDivis \charNummerVier {\kern 0.15ex}%0
% \XeTeXinterchartoks \charDivis \charNummerAcht {\kern 0.15ex}%0

\XeTeXinterchartoks \charDivis \charI {\kern 0.15ex}%0

% % Ausgleich vor Divis
\XeTeXinterchartoks \chard \charDivis {\kern 0.1ex}%0
\XeTeXinterchartoks \charf \charDivis {\kern 0.25ex}%0

\XeTeXinterchartoks \charA \charDivis {\kern 0.04ex}%0
\XeTeXinterchartoks \charI \charDivis {\kern 0.15ex}%0

\XeTeXinterchartoks \charn \charn {\kern-0.1ex}

% % Ausgleich vor Kürzungszeichen
\XeTeXinterchartoks \charf \charKzg {\kern1pt}

% % Ausgleich hinter Kürzungszeichen
\XeTeXinterchartoks \charKzg \chark {\nobreak\hskip-0.25pt\relax}
\XeTeXinterchartoks \charKzg \charn {\nobreak\hskip-0.5pt\relax}
\XeTeXinterchartoks \charKzg \charo {\nobreak\hskip-0.3pt\relax}
\XeTeXinterchartoks \charKzg \charu {\nobreak\hskip-0.5pt\relax}

%%% Character Protrusion für Junicode laden (geht automatisch nicht ohne Probleme)
% einige zusätzliche Werte definiert (spitze supplied-Klammern, Expansion)
\DeclareCharacterInheritance
   { encoding = {TU} }
   { A = {À,Á,Â,Ä,Å},
     F = {Ḟ},
     f = {ḟ},
     J = {Ĵ},
     K = {Ķ,Ǩ,Ḱ,Ḳ,Ḵ},
     L = {Ł,Ļ,Ľ,Ḷ,Ḹ,Ḻ,Ḽ},
     r = {ŕ,ŗ,ř,ȑ,ȓ,ṙ,ṛ,ṝ,ṟ},
     T = {Ţ,Ť,Ț,Ṫ,Ṭ,Ṯ,Ṱ},
     t = {ţ,ț,ṫ,ṭ,ṯ,ṱ,ẗ},
     V = {Ṽ,Ṿ},
     v = {ṽ,ṿ},
     W = {Ŵ,Ẁ,Ẃ,Ẅ,Ẇ,Ẉ},
     w = {ŵ,ẁ,ẃ,ẅ,ẇ,ẉ,ẘ},
     X = {Ẋ,Ẍ},
     x = {ẋ,ẍ},
     Y = {Ý,Ŷ,Ÿ,Ȳ,Ẏ,Ỳ,Ỵ,Ỷ,Ỹ},
     y = {ý,ÿ,ŷ,ȳ,ẏ,ỳ,ỵ,ỷ,ỹ},
     : = {/colon.alt},
     ; = {/semicolon.alt},
     ! = {/exclam.alt},
     ? = {/question.alt},
     ‘ = {/quoteleft.alt,/quotedblleft.alt},
     ’ = {/quoteright.alt,/quotedblright.alt}
   }

\SetProtrusion
   [ name     = Junicode-default,
     unit     = 1em ]
   { encoding = {EU1,EU2,TU},
     family   = Junicode }
   {
     A = {34,34},
     Æ = {69,  },
     F = {  ,29},
     J = {18,  },
     K = {  ,33},
     L = {  ,33},
     S = {33,15},
     T = {33,33},
     V = {63,31},
     W = {63,31},
     X = {33,33},
     Y = {30,30},
     f = {  ,-23},
     g = {  ,-9},
     k = {  ,25},
     r = {  ,7},
     t = {  ,6},
     v = {21,21},
     w = {21,21},
     x = {23,23},
     y = {  ,21},
     1 = {105,91},
     4 = {24,59},
     7 = {24,24},
     . = { ,178},
    {,}= { ,153},
     : = { ,146},
     ; = { ,87},
     ! = { ,26},
     ? = { ,37},
     @ = {37,37},
     ~ = {66,83},
    \% = {45,45},
     * = {79,79},
     + = {125,125},
     ( = {78,   },
     ) = {   ,69},
     / = {35,105},
     - = {280,280}, % hyphen
     –  = {280,280}, % endash
     — = {280,280}, % emdash
     ‘ = {96,76},
     ’ = {76,96},%〉 = { , 150},
     〉 = { , 200},% supplied
     ˈ = { , 200},% Expansion
     2 = { , 500}
   }
\SetProtrusion
   [ name     = Junicode-it,
     unit     = 1em ]
   { encoding = {EU1,EU2,TU},
     family   = Junicode,
     shape    = {it,sl}  }
   {
     A = {43,  },
     Æ = {70,  },
     T = {43,  },
     f = {  ,-35},
    {,}= { ,153},
     : = { ,110},
     ; = { ,110},
     ? = { ,66},
     & = {  ,62},
    \% = {47,47},
     / = {  ,40},
     - = {41,142}, % hyphen
     – = {41,142}, % endash
     — = {41,142}, % emdash
     ‘ = {84,105},
     ’ = {84,105},
     ( = {66,   },
     ) = {   ,66}
   }
%%% ENDE Character Protrusion
